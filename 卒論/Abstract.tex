% !TEX root = thesis.tex
% !TEX encoding = Shift_JIS
% !TEX spellcheck = none


製造業分野では少品種大量の製品を高品質にライン生産してきたが,2000年代に入って多様化する市場ニーズに応えるため,人の知能をより積極的に援用する人セル生産方式を導入した.しかし人セル生産方式の生産能力はライン方式に比して低位で,作業者スキル由来の品質ばらつきは直近の課題であり,中長期的には生産人口減少問題と無縁ではない.そこで本研究では,ライン方式と人セル方式のそれぞれのメリットを集めたセル生産ロボット方式を実現するためのロボット知能化技術に取り組んだ.そのなかで知能としてとりわけ自律性が重要なことがわかった.よって本論文では製造業向けロボットに自律性を付与するシステムアルゴリズムを開発し,以下の結果を得たことを論述する.

まず,人の自律性を特徴づける試行錯誤による習熟を,未知の目的関数の最適化問題と定式化して解く独自の「能動型探索アルゴリズム」を提案する.これは,その問題設定と,その試行で得られるであろう情報量・情報が得られないリスク量を同時に考慮して次の試行を最適化する点とに独創性がある.実証実験の結果,熟練者の調整で動作時間1045.33[ms]のロボット軌道が,約56\%の590.22[ms]に短縮したこと,さらに熟練者による加工機制御パラメータ調整などの同種の困難性を持つ問題への水平展開に成功したことで,有用性が示される.

次に,組立工程で切望されるバラ積み状態の部品供給に取り組む.これはロボット学にとってランダム・ビン・ピッキングという未解決の古典的難題である.この問題を複数の小問題に分割して解くシステム解法と,作業計画の試行錯誤を実行する「持ち替えグラフアルゴリズム」とを提案する.これらは既に有りそうで実は無かった方法である点,実行可能解を網羅的に算出する点に独創性がある.実証システムにおいて,数[g]から数10[g]の質量の実際の小型電機製品を構成する11種類以上の金属およびプラスチックの部品を,最速2秒台後半の時間間隔にて指定された姿勢に整列させられ,新規部品の追加に要するエンジニアリング時間は部品一つ当たり0.5日であることから,有用性がある.

\endinput
