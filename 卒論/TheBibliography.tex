% !TEX root = thesis.tex
% !TEX encoding = Shift_JIS
% !TEX spellcheck = none
% 「Reference.bib」を使う場合はこのファイルは使用しない
%
\begin{thebibliography}{99}
\addcontentsline{toc}{section}{参考文献}
\bibitem{Babu} S.S.Babu et al.: ``Empirical model of effects of pressure and temperature on electrical contact resistance of metals,'' Science and Technology of Welding and Joining. vol. 6, no. 12, pp. 126-132, 2001. 
\bibitem{niho} 二保 知也, ほか: ``微視接触電気抵抗解析に基づく抵抗スポット溶接のマルチスケール連成シミュレーション'', 第28回計算力学講演会文集, no. 15-19, 2015. 
\bibitem{Holo} R. Holm: ``Electric Contacts,'' Springer-Verlag, 1958. 
\bibitem{Greenwood} J.A.Greenwood: ``Constriction resistance and the real area of contact,'' British Journal of Applied Physics, vol. 17, no. 3, pp. 1621-1632, 1966. 
\bibitem{OpenCV} Gary Bradski, Adrian Kaehler: ``詳細OpenCV コンピュータビジョンライブラリを使った画像処理・認識'', 株式会社オーム社, 2012, pp. 195-223. 
\bibitem{iroha} 一色 実, 三村 耕司, 鈴木 茂: ``高純度鉄の精製と表面酸化挙動'', 真空, 45, 5, 2002, pp. 395-401. 
\bibitem{samsof} 津田 惟雄, ほか: ``物理科学選書 電気伝導性酸化物(改訂版)'', 裳華房, 1993, p. 26. 
\bibitem{honma} 本間 禎一: ``酸化皮膜の機械的性質と金属の酸化'', 防食技術, 25, 4, 1976, pp. 251-265. 
\bibitem{oya} 倉前 宏行, ほか: ``マルチスケール抵抗スポット溶接解析のための微視有限要素モデルの検討'', 第30回計算力学講演会論文集, 2017. 
\end{thebibliography}

\section*{本研究に関する発表リスト}
\addcontentsline{toc}{section}{本研究に関する発表}
\begin{enumerate}
\renewcommand{\labelenumi}{[\arabic{enumi}]}
\item
倉前宏行, 二保知也, 荒牧弘親, 大矢健吾, 堀江知義:マルチスケール抵抗スポット溶接解析のための微視有限要素モデルの検討, 日本機械学会 第30回計算力学講演会, 近畿大学東大阪キャンパス(大阪府東大阪市), 2017年9月16日

\item
倉前宏行, 大矢健吾:抵抗スポット溶接のマルチスケール・マルチフィジックス有限要素シミュレーション, 若手研究室見本市「イノベーションOIT 2017」, 大阪工業大学大宮キャンパス, 2017年10月26日
\end{enumerate}




\endinput
