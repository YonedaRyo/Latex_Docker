%パッケージ定義など
\documentclass[12pt,a4j, titlepage]{jarticle}
\usepackage{amsmath,amssymb}
\usepackage{bm}
\usepackage{here}
\usepackage[dvipdfmx]{graphicx}
\usepackage[dvipdfmx]{color}
\usepackage{siunitx}
\usepackage[hyphens]{url}
\usepackage{color}
\usepackage{datetime}
%\usepackage{datetime2} 
%
%\DTMcurrenttime←現在時刻(datetime2で使える)
%
\makeatletter%%%%%%%%%%%%%%%%%%%%%%%%%%%%%%%%%%%%%%%%%%参考文献に関する設定
\renewenvironment{thebibliography}[1]
{\section*{\refname\@mkboth{\refname}{\refname}}%
  \list{\@biblabel{\@arabic\c@enumiv}}%
       {\settowidth\labelwidth{\@biblabel{#1}}%
        \leftmargin\labelwidth
        \advance\leftmargin\labelsep
 \setlength\itemsep{-0.3zh}%←ここの数値を調整(行間のつまり具合)
 %\setlength\baselineskip{11pt}%←ここの数値を調整(追加)(文字の大きさ)
        \@openbib@code
        \usecounter{enumiv}%
        \let\p@enumiv\@empty
        \renewcommand\theenumiv{\@arabic\c@enumiv}}%
  \sloppy
  \clubpenalty4000
  \@clubpenalty\clubpenalty
  \widowpenalty4000%
  \sfcode`\.\@m}
 {\def\@noitemerr
   {\@latex@warning{Empty `thebibliography' environment}}%
  \endlist}
\makeatother%%%%%%%%%%%%%%%%%%%%%%%%%%%%%%%%%%%%%%%%%%%%%%
%%%%%%%%%%%%%%%%%%%%%%%%%%%%%%%%%%%%%%%%%%%%%%%%%%%%%%%%%%%%%%
%↓↓↓追加by宮堺
\newcommand{\Fig}{図}%←学会の様式に合わせる{Fig. }
\newcommand{\Tab}{表}%←学会の様式に合わせる{Table }
%%%%%%%%%%%%%%%%%%%%%%%%%%%%%%%%%%%%%%%%%%%%%%%%%%%%%%%%%%%%%%
%項目
\newcommand{\Introduction}{緒言}
\newcommand{\PreviousResearch}{先行研究}
\newcommand{\SuggestionsIssues}{提案手法とその課題}
\newcommand{\Suggestions}{提案内容}
\newcommand{\Issues}{課題}
\newcommand{\RelatedResearch}{関連研究}
\newcommand{\Solution}{解決方法}
\newcommand{\SolutionPartsIdentification}{部品種判定法}
\newcommand{\SolutionEliminatesDifferencesBetweenDifferentEnvironments}{異環境差除去法}
\newcommand{\SolutionCoordinateTransformation}{座標変換}
\newcommand{\SolutionQR}{二次元コード検出}
\newcommand{\SolutionAR}{ARマーカ検出}
\newcommand{\Environment}{想定環境と対象ワーク}
\newcommand{\Recognition}{組立工程認識処理}
\newcommand{\RecognitionProg}{組立工程認識プログラム}
\newcommand{\RecognitionExp}{評価実験}
\newcommand{\RecognitionExpConditions}{実験条件}
\newcommand{\RecognitionExpResults}{実験結果}
\newcommand{\RecognitionConsideration}{考察}
\newcommand{\Reproduction}{組立工程再現処理}
\newcommand{\ReproductionProg}{組立工程再現プログラム}
\newcommand{\ObjRecogAlgorithm}{物体認識アルゴリズム}
\newcommand{\RotationOperationDecisionAlgorithm}{回転操作判断アルゴリズム}
\newcommand{\ReproductionExp}{評価実験}
\newcommand{\ReproductionExpConditions}{実験条件}
\newcommand{\ReproductionExpResults}{実験結果}
\newcommand{\ReproductionConsideration}{考察}
\newcommand{\Conclusion}{結言}
\newcommand{\Summary}{まとめ}
\newcommand{\SummaryRecognition}{組立工程認識処理}
\newcommand{\SummaryReproduction}{組立工程再現処理}
\newcommand{\UpcomingDevelopments}{今後の展開}
\newcommand{\UpcomingDevelopmentsRecognition}{組立工程認識処理}
\newcommand{\UpcomingDevelopmentsReproduction}{組立工程再現処理}
%
\endinput
