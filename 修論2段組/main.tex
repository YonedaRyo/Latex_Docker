% !TeX encoding = Shift_JIS
% !TeX root = abstract.tex
% !TeX spellcheck = none

%\documentclass[uplatex,a4paper]{jarticle}
%\documentclass[a4paper]{ujarticle}
\documentclass[a4paper]{jarticle}
%\usepackage{master_abstract}
\usepackage{master_abstract_noeng}
\usepackage{amsmath}
\usepackage{here}

\usepackage[dvipdfmx]{color}
\usepackage{graphicx}
\usepackage[subrefformat=parens]{subcaption}
\usepackage{bm}
\usepackage{url}
%\usepackage{verbatim}
%\usepackage{multicol}
\usepackage{listings}
\usepackage{geometry}
\usepackage{multirow}
%\usepackage{xcolor}
\usepackage{cite}
\usepackage{comment}
%\usepackage[noto-otc]{pxchfon}  % フォッEト関??%\usepackage[deluxe]{otf}    % 多書体??B????
\geometry{left=20mm,right=20mm,top=25mm,bottom=22mm}
\setlength{\textwidth}{170mm}
\setlength{\textheight}{252mm}

\begin{document}

\id{M19-R26} 
\title{和文タイトルはこんな感じの研究} 
\name{工梅ロボ太郎} 
\tname{小林裕}
\etitle{English Title for This Research} 
\ename{KOUUME Robotarou}
% \abst{
%A manuscript should be prepared with a laser printer.  You must submit A4 sheets with a 
%top margin of 25mm, left and right margins of 20mm, and a bottom margin of 22mm. 
%The title and the name of the author should be printed on the first page in both Japanese 
%and English, followed by the abstract of 200-300 words (if possible), giving a brief 
%account of the most relevant aspects of the paper. Each paper must include five to ten 
%keywords in order to indicate the main topics discussed in the paper and to provide basic 
%terms for indexing. Main text will start with a line spacing above.  All figures and 
%tables are positioned within text.
%}
%\keywd{Mechanical engineering, Original paper, Guideline for manuscrip}


% \begin{figure}[H]
%    \centering
%    \includegraphics[width=0.53\hsize]{figure.eps}
%    \caption{Relationship between da/dn and $\Delta$ K.}
%    \label{fig:figure}
%\end{figure}

 \maketitle

%
% !TeX encoding = Shift_JIS
% !TeX root = ../abstract.tex
% !TeX spellcheck = none

\section{はじめに}
htふぇdzhbzせl;kbm:ぃkzdjf:んぴあzjんg:bp;じおkjrfg:psじおjg:pszぢfjgkfklんkkkkkgkgkgkkkgkgkkkkkkkkkkkkkkkkkkkkkkkkkkkkkkkkkkkkkkkkkkkkkkkkkkkkkkkkkkkkkkkkk
はじめにはじめにはじめにはじめにはじめにはじめにはじめにはじめにはじめにはじめに


\begin{comment}

\input{chapter2/abst_chapter2.tex}
\input{chapter3/abst_chapter3.tex}
\input{chapter4/abst_chapter4.tex}
\input{chapter5/abst_chapter5.tex}
\input{chapter6/abst_chapter6.tex}
\end{comment}

%\input{abst_bibliography.tex}

\end{document}
